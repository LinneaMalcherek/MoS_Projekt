% !TEX TS-program = pdflatex
% !TEX encoding = UTF-8 Unicode

% This is a simple template for a LaTeX document using the "article" class.
% See "book", "report", "letter" for other types of document.

\documentclass[11pt]{article} % use larger type; default would be 10pt

\usepackage[utf8]{inputenc} % set input encoding (not needed with XeLaTeX)

%%% Examples of Article customizations
% These packages are optional, depending whether you want the features they provide.
% See the LaTeX Companion or other references for full information.

%%% PAGE DIMENSIONS
\usepackage{geometry} % to change the page dimensions
\geometry{a4paper} % or letterpaper (US) or a5paper or....
% \geometry{margin=2in} % for example, change the margins to 2 inches all round
% \geometry{landscape} % set up the page for landscape
%   read geometry.pdf for detailed page layout information

\usepackage{graphicx} % support the \includegraphics command and options
\usepackage{datetime}
\usepackage{amssymb,amsmath} % For mathematical expressions  (Tillagd av Nai) 

%%% NEW COMMANDS
\renewcommand{\dateseparator}{-}
\newdateformat{mydate}{\THEYEAR \dateseparator0\THEMONTH \dateseparator \THEDAY} 
\renewcommand{\figurename}{Figur}

% \usepackage[parfill]{parskip} % Activate to begin paragraphs with an empty line rather than an indent

%%% PACKAGES
\usepackage{booktabs} % for much better looking tables
\usepackage{array} % for better arrays (eg matrices) in maths
\usepackage{paralist} % very flexible & customisable lists (eg. enumerate/itemize, etc.)
\usepackage{verbatim} % adds environment for commenting out blocks of text & for better verbatim
\usepackage{subfig} % make it possible to include more than one captioned figure/table in a single float
% These packages are all incorporated in the memoir class to one degree or another...

%%% HEADERS & FOOTERS
\usepackage{fancyhdr} % This should be set AFTER setting up the page geometry
\pagestyle{fancy} % options: empty , plain , fancy
\renewcommand{\headrulewidth}{0pt} % customise the layout...
\lhead{}\chead{}\rhead{}
\lfoot{}\cfoot{\thepage}\rfoot{}

%%% SECTION TITLE APPEARANCE
\usepackage{sectsty}
\allsectionsfont{\sffamily\mdseries\upshape} % (See the fntguide.pdf for font help)
% (This matches ConTeXt defaults)

%%% ToC (table of contents) APPEARANCE
\usepackage[nottoc,notlof,notlot]{tocbibind} % Put the bibliography in the ToC
\usepackage[titles,subfigure]{tocloft} % Alter the style of the Table of Contents
\renewcommand{\cftsecfont}{\rmfamily\mdseries\upshape}
\renewcommand{\cftsecpagefont}{\rmfamily\mdseries\upshape} % No bold!

%%% END Article customizations






%%% The "real" document content comes below...

\pagenumbering{gobble}
\title{Projektrapport Grupp 3 \\* 
CUURRLLINNNGGG!\\*
TNM??? Modelleringsprojekt}
\author{Linnéa Mellblom\\*Linnea Malcherek\\* Julia Nilsson\\*Michael Nilsson\\*Linnéa Nåbo}
%\date{} % Activate to display a given date or no date (if empty),
         % otherwise the current date is printed 
\mydate


\begin{document}
\maketitle
\pagebreak
\pagenumbering{arabic}  

\section{Redogörelse för arbetet}

\subsection{Translation}

I beräkningarna av stenens rörelse har hänsyn tagits till tre påverkande krafter: Kraft i stenens rörelseriktning, samt två krafter i ortogonal riktning mot denna \eqref{Ftot}. Dessa två krafter utgörs av friktionskrafterna i främre delen av stenen samt i den bakre delen. Differensen mellan dessa två krafter är vad som påverkar stenens curl. 
Stenens curl beror på att friktionen i den bakre delen av stenen är högre än den främre, på grund av de spår som den främre delen av stenen skapar i isen. REFERENS! 

 \begin{align}\label{Ftot}
 \bar{F_t}&=\bar{F_f}+\bar{F_b}+\bar{F}\\
 \bar{F_f}&<\bar{F_b}
 \end{align}

Translationen av stenen är en resulterande hastighetsvektor v, som består av hastigheten i rörelseriktningen samt av hastigheten i punkterna på det ringformade band stenen roterar på. I beräkningarna har dessa förenklats till två hastighetsvektorer: en för den främsta punkten (p1)  på stenens band och en för den bakersta (p2),(Figur~\ref{fig:Translation}). Den resulterande hastigheten består i beräkningarna således av tre komponenter  \eqref{vtot}. 

 \begin{align}\label{vtot}
 &\bar{v}=\bar{v_f}+(\bar{v_1}+\bar{v_2})
 \end{align}

Berkäkningen av hastigheterna i punkterna p1 och p2 beräknas i två steg. I första steget omvandlas stenens rotationshastighet till translationshastighet i riktningen i punkternas rotationsriktning. I det andra steget beräknas den påverkan som friktionen har på hastigheten i de två punkterna. Resultatet av detta blir två hastighetsvektorer i motsatta riktningar där den ena är större än den andra och resultanten blir således den riktning åt vilken stenen curlar. 

 \begin{align}\label{vsida_init}
 v_1 = v_1 = \omega r \\
 \end{align}

Därefter beräknas den nya hastigheten i dessa två punkter då de påverkats av de olika friktionerna, där den negativa accelerationen påverkas av hastigheten i punkten. 

 \begin{align}\label{vsida_init}
 v_1& = \\
v_2& = \\
 \end{align}

Translationen av stenen i riktning framåt beräknas enligt Eulers stegmetod och med konstant friktionskonstant


 \begin{align} % &-tecknen gör att de alignar
 \mu&=\frac{c}{\sqrt{v}}\\
% Position
 pos&=pos_n+\bar{v_n}\Delta t \\  %Oklar på hur man gör längre nedsänkningar än ett tecken (ex. n+1) 
% Speeds and velocities
 v&=v_n+a \Delta t \\ 
 \omega&=\omega_n+\alpha \Delta t  
 \end{align}



\section{Mål}



\begin{enumerate}
\item Här är en numrerad lista
\item Filtrera 
\item Kontrollera 
\item Göra 
\item Skapa 
\item Möjlighet 
\end{enumerate}


\pagebreak 



\subsection{Kravhantering}

Här är en sektion

\subsection{Principer och rutiner vid testning}

\subsubsection{Allmänt}
Här är en subsektion

\subsubsection{Rutiner för test}
Och en subsub!

Här är en onumrerad lista
\begin{itemize}
\item Olika
\item Att 
\item Att
\item Om 
\end{itemize}



Systemarkitekturen utgörs av ett Use-Case Diagram, och här kommer den automatiska referensen till den:   (Figure ~\ref{fig:Translation}).

\begin{figure}[ht!]
\centering
\includegraphics[width=80mm]{Translation.png}
\caption{Translation}
\label{fig:Translation}
\label{overflow}
\end{figure}


\pagebreak


\section{Projektmedlemmar}

Linnea Malcherek - projektledare och scrum master\\*
Julia Nilsson - kundkontakt och produktägare\\*
Linnéa Nåbo \\*
Lovisa Dahl\\*





\end{document}
